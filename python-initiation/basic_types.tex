\section{Les types de base de Python}
\subsection{Quelques types simples}
% 1 Les types de base Python
% 1.1 Quelques types simples
%     - entiers
%     - les réels (flottants)
%     - les complexes
%     - les booléens (True, False)
%     - le type None

\begin{frame}
  \frametitle{Quelques types simples}
  \begin{itemize}
    \item<1-> les entiers
    \item<2-> les flottants
    \item<3-> les complexes
    \item<4-> les booléens
    \item<5-> la valeur 'Rien'
  \end{itemize}
\end{frame}

\begin{frame}[fragile]
  \frametitle{Quelques types simples}
  \begin{itemize}
    \item les entiers
  \end{itemize}
\begin{ipython}
\ipinpt{my_int = 12}
\ipinpt{my_other_int = int("12")}
\ipinpt{my_int + my_other_int}
\ipoutp{24}
\end{ipython}
\end{frame}

\begin{frame}[fragile]
  \frametitle{Quelques types simples}
  \begin{itemize}
    \item les flottants
  \end{itemize}
\begin{ipython}
\ipinpt{my_float = 3.14}
\ipinpt{print float(42)}
\ipoutp{42.0}
\ipinpt{print 2/5}
\ipoutp{0}
\ipinpt{print 2/5.0}
\ipoutp{0.40000000000000002}
\end{ipython}
\end{frame}

\begin{frame}[fragile]
  \frametitle{Quelques types simples}
  \begin{itemize}
    \item les complexes
  \end{itemize}
\begin{ipython}
\ipinpt{my_cplx = 5+2j}
\ipinpt{my_cplx.real}
\ipoutp{5.0}
\ipinpt{my_cplx.imag}
\ipoutp{2.0}
\ipinpt{my_cplx.conjugate()}
\ipoutp{(5-2j)}
\ipinpt{abs(my_cplx)}
\ipoutp{5.3851648071345037}
\end{ipython}
\end{frame}

\begin{frame}[fragile]
  \frametitle{Quelques types simples}
  \begin{itemize}
    \item les booléens : \alert{True, False}
    \item la valeur 'Rien' : \alert{None}
  \end{itemize}
\begin{ipython}
\ipinpt{print True | False}
\ipoutp{True}
\ipinpt{print True and False}
\ipoutp{False}
\ipinpt{print type(None)}
\ipoutp{<type 'NoneType'>}
\end{ipython}
\end{frame}

% 1.2 Quelques structures de données
%     - les chaines de caractère sont des séquences comme les autres mais sont différentes dans la tête des gens
%     - séquence: list modifiable, tuple non modifiable, queue...
%     - dictionnaire
%     - ensemble (partie gauche d'un dict): set modifiable, frozenset non modifiable

\subsection{Quelques structures de données}
\begin{frame}
  \frametitle{Quelques structures de données}
  \begin{itemize}
    \item<1-> les chaînes de caractère, un cas particulier de séquence
    \item<2-> les autres séquences : tuples, listes \ldots
    \item<3-> les dictionnaires
    \item<4-> les sets, frozenset
  \end{itemize}
\end{frame}

\begin{frame}[fragile]
\frametitle{Quelques structures de données}
\begin{itemize}
\item Généralités sur les séquences (liste, tuple, chaînes)
\end{itemize}
\begin{ipython}
\ipinpt{sequence = [1,2,3,4] \#liste pour exemple}
\ipinpt{print sequence[1]}
\ipoutp{2}
\ipinpt{print sequence[:3]}
\ipoutp{[1,2,3]}
\ipinpt{print sequence[-1]}
\ipoutp{4}
\end{ipython}
\end{frame}

\begin{frame}[fragile]
  \frametitle{Quelques structures de données}
  \begin{itemize}
    \item les chaînes de caractères
  \end{itemize}
\begin{ipython}
\ipinpt{my\_str = "hello"}
\ipinpt{my\_other\_str = 'hello'}
\ipinpt{my\_third\_str = """My}
\ipcont{tailor is}
\ipcont{rich."""}
\ipinpt{my\_last\_str = str(42)}
\ipinpt{my\_last\_str + my\_str}
\ipoutp{42hello}
\end{ipython}
\end{frame}

\begin{frame}[fragile]
  \frametitle{Quelques structures de données}
    \begin{itemize}
    \item manipulation de chaîne de caractères
    \end{itemize}
\begin{ipython}
\ipinpt{a = "une string"}
\ipinpt{print(a.split(' '))}
\ipoutp{['une', 'string']}
\ipinpt{print a.upper()}
\ipoutp{UNE STRING}
\end{ipython}
\end{frame}

\begin{frame}[fragile]
  \frametitle{Quelques structures de données}
  \begin{itemize}
    \item les listes
  \end{itemize}
  \begin{ipython}
\ipinpt{my\_str = "cardboard"}
\ipinpt{my\_list = [1, my\_str, (2, 3, 4)]}
\ipinpt{my\_list.append(321)}
\ipinpt{print my\_list}
\ipoutp{[1, 'cardboard', (2, 3, 4), 321]}
\ipinpt{my\_list.extend([2,3])}
\ipinpt{print my\_list}
\ipoutp{[1, 'cardboard', (2, 3, 4), 321, 2, 3]}
  \end{ipython}
\end{frame}

\begin{frame}[fragile]
  \frametitle{Quelques structures de données}
    \begin{itemize}
      \item les tuples (séquences non éditables)
    \end{itemize}
\begin{ipython}
\ipinpt{my\_tuple = (1, 1, "hello", 1, 1.5)}
\ipinpt{print my\_tuple.count(1)}
\ipoutp{3}
\ipinpt{my\_other\_tuple = tuple("epsilon")}
\ipinpt{my\_other\_tuple}
\ipoutp{('e', 'p', 's', 'i', 'l', 'o', 'n')}
\end{ipython}
\end{frame}

\begin{frame}[fragile]
  \frametitle{Quelques structures de données}
  \begin{itemize}
    \item les dictionnaires (tableaux associatifs)
  \end{itemize}
  \begin{ipython}
\ipinpt{my\_dict = \{"key1":"value1", }
\ipcont{           "key2":"value2"\}}
\ipinpt{my\_dict["key3"] = "value3"}
\ipinpt{my\_dict["key2"]}
\ipoutp{'value2'}
\ipinpt{my\_dict.keys()}
\ipoutp{['key3', 'key2', 'key1']}
\ipinpt{my\_dict.items()}
\ipoutp{[('key3', 'value3'), ('key2', 'value2'),}
\ipwrapp{('key1', 'value1')]}
  \end{ipython}
\end{frame}

\begin{frame}[fragile]
  \frametitle{Quelques structures de données}
  \begin{itemize}
    \item Les sets assurent l'unicité de leurs éléments
    \item Les sets permettent des opérations ensemblistes (union, différence symétrique)
    \item Seuls des éléments 'hashable' peuvent être insérés dans les sets
    \item Les frozensets sont aux sets ce que le tuple est au liste (non éditable)
  \end{itemize}
\end{frame}

\begin{frame}[fragile]
  \frametitle{Quelques structures de données}
  \begin{ipython}
\ipinpt{my\_set = set(["blue", "green", "green"])}
\ipinpt{my\_set.add("yellow")}
\ipinpt{my\_set}
\ipoutp{set(['blue', 'green', 'yellow'])}
\ipinpt{my\_set.add("yellow")}
\ipinpt{my\_set}
\ipoutp{set(['blue', 'green', 'yellow']])}
\ipinpt{my\_other\_set = set(("green", "rogen"))}
\ipinpt{my\_set.intersection(my\_other\_set)}
\ipoutp{set(['green'])}
  \end{ipython}
\end{frame}

% 1.3 Quelques autres types courants (juste les évoquer)
%     - classe, fonction ou méthode, module...

\subsection{Quelques autres types courants}
\begin{frame}
  \frametitle{Quelques autres types courants}
  \begin{itemize}
    \item les classes
    \item les fonctions/méthodes
    \item les modules
  \end{itemize}
\end{frame}

% 2 La syntaxe de Python
% 2.1 instructions
%     - séparateur d'instruction : ';' (éviter) ou '\\n'
% 2.2 blocs
%     - contexte (scope) défini par bloc
%     - blocs définis par :
%       - leur niveau d'indentation
%       - bloc de niveau inférieur suit une ligne
%         - qui commence par un mot clef (if, while, for, def, class, with)
%         - se termine par ':'
% et continuer comme julien l'a fait sur if etc.
% Ce point 2.2 est une introduction aux différents blocs qu'on va traiter ensuite.
