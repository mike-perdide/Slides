\section{Python : origine et philosophie}
\subsection{L'origine de Python}
\begin{frame}
  \frametitle{Python - L'origine}
  \begin{columns}
    \begin{column}{5cm}
      \begin{itemize}
        \item<1-> Apparition en 1991
        \item<2-> Créé par Guido van Rossum
        \item<3-> Nombreuses références aux Monty Python
      \end{itemize}
    \end{column}
    \mode<presentation>{
    \begin{column}{5cm}
      \begin{overprint}
        \includegraphics<2>[scale=0.04]{medias/guido.jpg}
        \includegraphics<3>[scale=0.15]{medias/spam.jpg}
      \end{overprint}
    \end{column}
    }
  \end{columns}
\end{frame}

\subsection{Valeurs et philosophie}
\begin{frame}
\frametitle{Python - Valeurs et philosophie}
  % En fait ici on présente ce qu'on va voir, afin de pouvoir terminer par CQFD!
  \begin{itemize}
    \item Des standards de codage indispensables
    \mode<article>{\linebreak Outre le caractère esthétique des standards de codage, ceux-ci garantissent également la portabilité du code sur d'autres plateformes.}
    \item Pas de magie
    \mode<article>{\linebreak L'interpréteur Python ne fait pas de magie, une erreur sera préférée à un comportement étrange.}
    \item Programmation souple : orientée aspect, haut niveau
    \mode<article>{\linebreak TODO}
    \item Convergence de l'élégance
    \mode<article>{\linebreak La philosophie du langage Python et des règles qui l'accompagnent tendent à programmer de manière cohérente et lisible emmenant par là même une maintenabilité accrue.}
  \end{itemize}
\end{frame}

\subsection{La recherche du meilleur chemin}
\begin{frame}
\frametitle{Python - La recherche du meilleur chemin}
  \begin{itemize}
    \item Un travail de recherche via les PEP (Python Enhancement Proposal). Exemples:
    \begin{itemize}
      \item PEP 8: Style Guide for Python Code
      \item PEP 100: Python Unicode Integration
      \item PEP 386: Changing the version comparison module in Distutils
    \end{itemize}
    \pause
    %note feth:
    %mots clefs: efficacité en terme de lisibilité, pythonique, élégance (comme en maths)
    \item Un seul bon moyen de faire, pas de comportement implicite
    \begin{itemize}
      \item Si une déclaration n'est pas assez explicite, on préferera faire échouer la commande
      \item "In the face of ambiguity, refuse the temptation to guess."
    \end{itemize}
    \pause
    \item Si le code n'est pas élégant, c'est qu'il y a une meilleure façon de faire
  \end{itemize}
\end{frame}

\begin{frame}[fragile]
\frametitle{En résumé, la philosophie de Python c'est :}
Tiré de la PEP 20, The Zen of Python :
  \begin{itemize}
    \item Beautiful is better than ugly.
    \item Explicit is better than implicit.
    \item Simple is better than complex.
    \item Complex is better than complicated.
    \item Flat is better than nested.
    \item Sparse is better than dense.
    \item Readability counts.
    \item Special cases aren't special enough to break the rules.
    \item There should be one-- and preferably only one --obvious way to do it.
    \item If the implementation is hard to explain, it's a bad idea.
  \end{itemize}
\end{frame}
\newpage
