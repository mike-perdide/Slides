\section{Introduction}

\begin{frame}
\frametitle{Objectifs de la formation}
  \begin{itemize}
  \item Présenter le langage et ses principes
  \item Décrire les grandes caractéristiques du langage et de sa syntaxe
  \item Etre capable de développer ses premiers programmes en python
  \item Fournir les outils nécessaires à l'exploration du langage
  \item Orienter vers les axes primordiaux
  \end{itemize}
\end{frame}

\mode<article>{
\begin{frame}
\frametitle{A qui s'adresse ce cours}
Ce cours est destiné à être une introduction basique à Python, il s'adresse à tous développeur s'intéressant au langage.
Il est préférable que vous ayez des notions de programmation afin de saisir le contenu de ce cours.
\end{frame}
}

\begin{frame}
\frametitle{Les spécificités de Python}
\begin{flushleft}
  \begin{itemize}
    \item Libre
    \mode<article>{\linebreak Le téléchargement et l'usage de Python peut se faire sans restriction.}
    \item Orienté objet
    \mode<article>{\linebreak Même si il est possible de faire de la programmation fonctionnelle en Python est bel et bien un langage objet. Les fonctions elles-mêmes sont en fait des objets}
    \item Portable
    \mode<article>{\linebreak Python est disponible pour l'ensemble des plateformes utilisée aujourd'hui. Si les règles sont bien respectées, un code Python ne nécessitera pas de modifications pour passer de Windows à Linux.}
    \item Interprété (pas de compilation manuelle)
    \mode<article>{\linebreak Votre code est directement utilisable, il n'est pas nécessaire de le compiler pour voir le résultat des 5 lignes que vous venez d'écrire.}
    \item Extensible (librairies standard, modules C, C++, \ldots)
    \mode<article>{\linebreak Python propose nativement une librairie standard fournissant un grand nombre de fonctionnalité (manipulation de fichier, réseau, log ...). Il existe un grand nombre de librairies annexes. Il est aisé de développer son propre module. Il est possible d'intégrer du code C ou C++ dans du code Python, il est possible d'invoquer du code Python depuis C, C++, Java.}
    \item Plaisant
    \mode<article>{\linebreak La facilité de lecture du code et le caractère explicite du langage rendent l'usage de Python plaisant. Après un certain temps, il n'est plus nécessaire de se référer systématiquement à la documentation.}
    \item Pratique (du script au programme)
    \mode<article>{\linebreak Python est tout à fait adapté au développement de programmes complexes (GUI, réseau, ...) mais permet également de développer des scripts d'administration de manière très efficientes.}
    \end{itemize}
    \end{flushleft}
\end{frame}
\newpage
