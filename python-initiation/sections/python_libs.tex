\section{Quelques librairies standard}
\begin{frame}[fragile]
  \frametitle{Quelques librairies standard}
  \begin{itemize}
    \item<1-> {\bf subprocess}, la gestion des process
    \item<2-> {\bf socket}, pour la communication réseau
    \item<3-> {\bf os}, interaction avec le système hôte
    \item<4-> {\bf threading}, pour les threads
  \end{itemize}
\end{frame}

\begin{frame}[fragile]
  \frametitle{Syntaxe de import}
  \begin{lstlisting}
import subprocess

subprocess.Popen(cmd)
  \end{lstlisting}

  \begin{lstlisting}
from os.path import isfile

isfile("/home/user/ipython/.ipythonrc")
  \end{lstlisting}

  \begin{lstlisting}
from threading import Thread as custom_name

class my_thread(custom_name):
    ...
  \end{lstlisting}
\end{frame}

\begin{frame}[fragile]
  \frametitle{Syntaxe de import}
Il est aussi possible d'utiliser la syntaxe :
  \begin{lstlisting}
from PyQt4.QtGui import *
  \end{lstlisting}
Mais c'est déconseillé, car cela peuple le namespace de manière implicite.
\end{frame}
\newpage
