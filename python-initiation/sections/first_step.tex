\section{Premiers pas}

\subsection{Qu'est-ce qu'un programme Python?}
\begin{frame}[fragile]
  \frametitle{Qu'est-ce qu'un process Python?}
  Comme pour tout programme :
  \begin{itemize}
    \item[3] Un interprêteur (processeur)
    \item[2] Un environnement d'exécution
    \item[1] Du code : ici, du bytecode
  \end{itemize}
\end{frame}

\begin{frame}[fragile]
  \frametitle{Le bytecode}
  \begin{itemize}
    \item Obtenu par \em{compilation} de code source
    \item Portable
  \end{itemize}
\end{frame}

\begin{frame}[fragile]
  \frametitle{L'environnement}
  Cette notion comprend
  \begin{itemize}
    \item Les variables d'environnement \pause
    \item Le système d'exploitation (OS)\ldots \pause
    \item Les IPC, fichiers, matériels\ldots auxquels l'OS donne accès \pause
    \item Via les périphériques matériels : l'utilisateur, l'univers
  \end{itemize}
  TODO: AJOUTER DES IMAGES?
\end{frame}

\begin{frame}[fragile]
  \frametitle{Le bytecode}
  \begin{itemize}
    \item Obtenu par \em{compilation} de code source
    \item Portable
  \end{itemize}
\end{frame}


\subsection{Au c\oe ur l'interprêteur Python}
\begin{frame}[fragile]
  \frametitle{Lancer Python interactivement}
  \begin{itemize}
    \item Shell python interactif
    \item Apporte en plus un certain nombre de fonctionnalités intéressantes
  \end{itemize}
\end{frame}
\newpage
