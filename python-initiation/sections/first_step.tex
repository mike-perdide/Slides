\section{Premiers pas}

\subsection{Qu'est-ce qu'un programme Python?}
\begin{frame}[fragile]
  \frametitle{Qu'est-ce qu'un process Python?}
  Comme pour tout programme :
  \begin{itemize}
    \item[1] Du code : ici, du bytecode
    \item[2] Un environnement d'exécution
    \item[3] Un interprêteur (processeur)
  \end{itemize}
\end{frame}

\begin{frame}[fragile]
  \frametitle{Le bytecode}
  \begin{llist}
    \item Obtenu par compilation de code source
    \mode<article>{\linebreak Le code source (fichier .py) est compilé en bytecode de manière transparente lors de l'appel à votre fichier.}
    \item Portable
  \end{llist}
\end{frame}

\begin{frame}[fragile]
  \frametitle{L'environnement d'éxécution}
  Cette notion comprend
  \begin{itemize}
    \item Les variables d'environnement
    \item Le système d'exploitation (OS)\ldots
    \item Les IPC, fichiers, matériels\ldots auxquels l'OS donne accès
    \item Via les périphériques matériels : l'utilisateur, l'univers
  \end{itemize}
  TODO: AJOUTER DES IMAGES?
\end{frame}

\subsection{Au c\oe ur de l'interprêteur Python}
\begin{frame}[fragile]
  \frametitle{Lancer Python interactivement}
  \begin{llist}
    \item Shell python interactif
    \item Apporte en plus un certain nombre de fonctionnalités intéressantes
  \end{llist}
\end{frame}

\newpage
%    \begin{llist}
%    \item La commande help
%    \mode<article>{\linebreak La commande help permet de découvrir l'usage et les spécificités d'un objet.}
%    \item La commande dir
%    \mode<article>{\linebreak La commande dir permet de lister les méthodes d'un objet}
%    \item Il est possible de modifier l'environnement.
%    \item \ldots
%    \end{llist}
