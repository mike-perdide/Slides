\section{Utiliser Python}

\subsection{Éxécution depuis un fichier}
\begin{frame}[fragile]
  \frametitle{Éxécution depuis un fichier}
  \begin{itemize}
  \item La déclaration de l'encodage
  \begin{lstlisting}
#-*-coding:utf-8-*-
  \end{lstlisting}
  \pause
  \item Les imports de module
  \begin{lstlisting}
import os
from sys import path
  \end{lstlisting}
  \item Les imports non explicites sont à proscrire
  \begin{lstlisting}
from sys import *
  \end{lstlisting}
  \end{itemize}
\end{frame}

\subsection{Script}
\begin{frame}[fragile]
  \frametitle{Script}
  \begin{itemize}
  \item Un test sur l'attribut \verb=__name__=  permet de transformer notre module en programme
  \begin{lstlisting}
if __name__ == '__main__':
    print("Programme de test pour mon module")
  \end{lstlisting}
  \pause
  \item Éxécution depuis un terminal
  \begin{beamercolorbox}{terminal}
\scriptsize\begin{verbatim}$ python monfichier.py
Programme de test pour mon module
$
\end{verbatim}
\end{beamercolorbox}
\end{itemize}
\end{frame}

\subsection{Import de mon module}
\begin{frame}[fragile]
  \frametitle{Import de mon module}
    \begin{itemize}
      \item Le fichier \verb=__init__.py= est nécessaire pour identifier le répertoire courant comme un package python
      \begin{lstlisting}
~/mypackage/__init__.py
~/mypackage/example.py
      \end{lstlisting}
      \pause
      \item Le package doit être dans le {\bf PYTHONPATH} pour pouvoir être importé
    \end{itemize}
\end{frame}

\begin{frame}[fragile]
\frametitle{Import de mon module}
\begin{ipython}
\ipinpt{from sys import path}
\ipinpt{print path}
\ipoutp{['', '/usr/bin', '/usr/local/lib/python2.6/dist}
\ipwrapp{-packages', ...]}
\ipinpt{sys.path.append('/home/user/')}
\ipinpt{from mypackage import example}
\ipinpt{print example}
\ipoutp{<module 'mypackage.example' from '/home/user/m}
\ipwrapp{ypackage/example.py'>}
\end{ipython}
\end{frame}
