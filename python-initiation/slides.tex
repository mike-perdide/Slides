\section{Python : origine et philosophie}

\begin{frame}
  \frametitle{L'origine de Python}
  \begin{columns}
    \begin{column}{5cm}
      \begin{itemize}
        \item<1-> Apparition en 1991
        \item<2-> Créé par Guido van Rossum
        \item<3-> Nombreuses références aux Monty Python
      \end{itemize}
    \end{column}
    \begin{column}{5cm}
      \begin{overprint}
        \includegraphics<2>[scale=0.04]{guido.jpg}
        \includegraphics<3>[scale=0.15]{spam.jpg}
      \end{overprint}
    \end{column}
  \end{columns}
\end{frame}

\begin{frame}
\frametitle{Valeurs et philosophie}
  \begin{itemize}
    \item Orienté objet
    \pause
    \item Extensible (librairies standard, modules C, C++, ...)
  \end{itemize}
\end{frame}

\begin{frame}
\frametitle{La recherche du meilleur chemin}
  \begin{itemize}
    \item Un travail de recherche via les PEP
    \pause
    \item 1 seul bon moyen de faire
  \end{itemize}
\end{frame}



\section{IPython, le Python interactif}

\begin{frame}
  \frametitle{Fonctionnalités}
  \begin{itemize}
    \item Exécution de code dynamique
    \item Interaction avec le système
    \item Historique des commandes
    \item Journalisation 
  \end{itemize}
\end{frame}

\begin{frame}
\frametitle{Utile pour ...}
  \begin{itemize}
    \item Apprentissage
    \item Découverte interactive d'une API
    \item Débug
    \item Shell IPython embarqué
  \end{itemize}
\end{frame}


%  \framesubtitle{The proof uses \textit{reductio ad absurdum}.}
%  \begin{theorem}
%      There is no largest prime number.
%  \end{theorem}
%  \begin{proof}
%    \begin{enumerate}
%      \item<1-| alert@1> Suppose $p$ were the largest prime number.
%      \item<2-> Let $q$ be the product of the first $p$ numbers.
%      \item<3-> Then $q+1$ is not divisible by any of them.
%      \item<1-> Thus $q+1$ is also prime and greater than $p$.\qedhere
%    \end{enumerate}
%  \end{proof}
