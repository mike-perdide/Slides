\section{Python : origine et philosophie}
\subsection{L'origine de Python}
\begin{frame}
  \frametitle{L'origine de Python}
  \begin{columns}
    \begin{column}{5cm}
      \begin{itemize}
        \item<1-> Apparition en 1991
        \item<2-> Créé par Guido van Rossum
        \item<3-> Nombreuses références aux Monty Python
      \end{itemize}
    \end{column}
    \begin{column}{5cm}
      \begin{overprint}
        \includegraphics<2>[scale=0.04]{guido.jpg}
        \includegraphics<3>[scale=0.15]{spam.jpg}
      \end{overprint}
    \end{column}
  \end{columns}
\end{frame}

\subsection{Valeurs et philosophie}
\begin{frame}
\frametitle{Valeurs et philosophie}
  \begin{itemize}
    \item Orienté objet
    \pause
    \item Extensible (librairies standard, modules C, C++, ...)
  \end{itemize}
\end{frame}

\subsection{La recherche du meilleur chemin}
\begin{frame}
\frametitle{La recherche du meilleur chemin}
  \begin{itemize}
    \item Un travail de recherche via les PEP
    \pause
    \item 1 seul bon moyen de faire
  \end{itemize}
\end{frame}



\section{IPython, le Python interactif}
\subsection{Les fonctionnalités}

\begin{frame}
  \frametitle{Fonctionnalités}
  \begin{itemize}
    \item Exécution de code dynamique
    \item Interaction avec le système
    \item Historique des commandes
    \item Journalisation 
  \end{itemize}
\end{frame}

\subsection{Utile pour ...}
\begin{frame}[fragile]
  \frametitle{Utile pour ...}
    \begin{itemize}
      \item apprendre la syntaxe
    \end{itemize}
  \includegraphics[scale=0.35]{apprendre.png}

\end{frame}

\begin{frame}
  \frametitle{Utile pour ...}
    \begin{itemize}
      \item prototyper une fonctionnalité
    \end{itemize}
  \includegraphics[scale=0.35]{prototype.png}
\end{frame}

\begin{frame}
  \frametitle{Utile pour ...}
    \begin{itemize}
      \item la découverte interactive d'une API
    \end{itemize}
  \includegraphics[scale=0.35]{api_discover.png}
\end{frame}

\begin{frame}
  \frametitle{Utile pour ...}
    \begin{itemize}
      \item embarquer un shell IPython dans ses programmes
    \end{itemize}
  \includegraphics[scale=0.35]{embedded_ipython.png}
\end{frame}

\section{La syntaxe Python}
\subsection{Les principaux types de structure}

\begin{frame}
  \frametitle{Les principaux types de structure}
    \begin{itemize}
      \item<1-> les entiers
      \item<2-> les chaînes de caractère
      \item<3-> les tuples
      \item<4-> les listes
      \item<5-> les sets
      \item<6-> les dictionnaires
      \item<7-> les booléens
      \item<8-> la valeur nulle
    \end{itemize}
\end{frame}

\begin{frame}
  \frametitle{Les principaux types de structure ...}
    \begin{itemize}
      \item les entiers
    \end{itemize}
    \includegraphics[scale=0.35]{type_int.png}
\end{frame}

\begin{frame}
  \frametitle{Les principaux types de structure ...}
    \begin{itemize}
      \item les chaînes de caractères
    \end{itemize}
    \includegraphics[scale=0.35]{type_str.png}
\end{frame}

\begin{frame}
  \frametitle{Les principaux types de structure ...}
    \begin{itemize}
      \item les tuples
    \end{itemize}
    \includegraphics[scale=0.35]{type_tuple.png}
\end{frame}

\begin{frame}
  \frametitle{Les principaux types de structure ...}
    \begin{itemize}
      \item les listes
    \end{itemize}
    \includegraphics[scale=0.35]{type_list.png}
\end{frame}

\begin{frame}
  \frametitle{Les principaux types de structure ...}
    \begin{itemize}
      \item les sets
    \end{itemize}
    \includegraphics[scale=0.35]{type_set.png}
\end{frame}

\begin{frame}
  \frametitle{Les principaux types de structure ...}
    \begin{itemize}
      \item les dictionnaires
    \end{itemize}
    \includegraphics[scale=0.35]{type_dict.png}
\end{frame}

\begin{frame}
  \frametitle{Les principaux types de structure ...}
    \begin{itemize}
      \item les booléens : True, False
      \item la valeur nulle : None
    \end{itemize}
\end{frame}


%  \framesubtitle{The proof uses \textit{reductio ad absurdum}.}
%  \begin{theorem}
%      There is no largest prime number.
%  \end{theorem}
%  \begin{proof}
%    \begin{enumerate}
%      \item<1-| alert@1> Suppose $p$ were the largest prime number.
%      \item<2-> Let $q$ be the product of the first $p$ numbers.
%      \item<3-> Then $q+1$ is not divisible by any of them.
%      \item<1-> Thus $q+1$ is also prime and greater than $p$.\qedhere
%    \end{enumerate}
%  \end{proof}

%\setbeamercolor{terminal}{fg=white, bg=black}
%\begin{beamercolorbox}{terminal}
%\end{beamercolorbox}
