\section{Python : origine et philosophie}
\subsection{L'origine de Python}
\begin{frame}
  \frametitle{L'origine de Python}
  \begin{columns}
    \begin{column}{5cm}
      \begin{itemize}
        \item<1-> Apparition en 1991
        \item<2-> Créé par Guido van Rossum
        \item<3-> Nombreuses références aux Monty Python
      \end{itemize}
    \end{column}
    \begin{column}{5cm}
      \begin{overprint}
        \includegraphics<2>[scale=0.04]{guido.jpg}
        \includegraphics<3>[scale=0.15]{spam.jpg}
      \end{overprint}
    \end{column}
  \end{columns}
\end{frame}

\subsection{Valeurs et philosophie}
\begin{frame}
\frametitle{Valeurs et philosophie}
  % En fait ici on présente ce qu'on va voir, afin de pouvoir terminer par CQFD!
  \begin{itemize}
    \item Orienté objet
    %note feth:
    %utilisation souple (programmation impérative, orientée aspect, haut niveau ou bas niveau...)
    \pause
    \item Extensible (librairies standard, modules C, C++, \ldots)
  \end{itemize}
\end{frame}

\subsection{La recherche du meilleur chemin}
\begin{frame}
\frametitle{La recherche du meilleur chemin}
  \begin{itemize}
    \item Un travail de recherche via les PEP
    \pause
    %note feth:
    %mots clefs: efficacité en terme de lisibilité, pythonique, élégance (comme en maths)
    \item un seul bon moyen de faire, pas de comportement par défaut comme en Perl par exemple
  \end{itemize}
\end{frame}
