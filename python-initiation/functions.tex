% Requires : ifthen,color,listings packages (maybe more)
% Build the box surrounding the ipython stuff (In : Out : ...)
\newcommand\ipythonbox[1]{
\makebox[33pt][l]{%
#1
}%
}

% Commands printing In[.]: Out[]: and breaklines :  ...: and \
\newcommand\ipythonin[1]{%
\setcounter{ipythoncolor}{0}%
\ipythonbox{%
\textcolor{green}{%
\textnormal{\textbf{In\space\space[#1]:}}%
}%
}%
}%

\newcommand\ipythonout[1]{%
\setcounter{ipythoncolor}{1}%
\ipythonbox{%
\textcolor{red}{%
\textnormal{\textbf{Out[#1]:}}%
}%
}%
}%

\newcommand\ipythondots{%
\ifthenelse{\equal{\theipythoncolor}{0}}{%
\ipythonbox{%
\textcolor{green}{%
\textnormal{\textbf{\space\space\space...:}}%
}%
}%
}{}%
}%

\newcommand\ipythonbackslash{%
\ifthenelse{\equal{\theipythoncolor}{0}}{%
\backslash}%
{}%
}%

\newcounter{inpy}%
\newcounter{uncovercount}%
\newcounter{ipythoncolor}%

% Definition du style de la console ipython
% TODO trouver comment virer les bordures blanches ???
\lstdefinestyle{styleLangage}{%
language           = python,%
basicstyle         = \footnotesize\ttfamily\color{white},% ecriture standard
identifierstyle    = \color{white},%
commentstyle       = \color{blue},%
keywordstyle       = \color{yellow},%
stringstyle        = \color{gray},%
emphstyle          = \color{green},%
extendedchars      = true,% permet d'avoir des accents dans le code
showspaces         = false,%
showstringspaces   = false,%
backgroundcolor    = \color{black},%
rulecolor          = \color{black},%
rulesepcolor       = \color{black},%
fillcolor          = \color{black},%
prebreak           = \hbox{$\ipythonbackslash{}$},%
postbreak          = \hbox{$\ipdots{}\space$},%
escapeinside       = ~~,%
breaklines         = true,%
breakatwhitespace  = true,
breakautoindent    = false,%
breakindent        = 0pt,
frame=none, %
columns=fixed,
lineskip=5pt%
}

\lstnewenvironment{ipython}[1]{%
\lstset{%
title={#1},
style = styleLangage%
}%
\setcounter{inpy}{0}%
\setcounter{uncovercount}{1}%
\setcounter{ipythoncolor}{0}%
}
{}

\newcommand\ipinpt[1]{%
\addtocounter{inpy}{1}%
\ipythonin{\theinpy}%
}

\newcommand\ipoutp[1]{%
\ipythonout{\theinpy}%
}
\newcommand\ipdots[1]{%
\ipythondots{}%
}
